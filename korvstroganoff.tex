\begin{recipe}[portion={\portion{4}},bakingtime={\SI{30}{\minute}}]{Korvstroganoff}
  \graph{big = korvstroganoff, bigpicturewidth = 0.3\textwidth}
  % \introduction{hejsan}
  \ingredients{%
    \SI{400}{g} & passerade tomater\\
    \SI{2.5}{dl} & matlagningsgrädde\\
    en & gul lök\\
    \SI{700}{g} & falukorv \\
    3 skedar&mjöl\\
    mycket&paprikapulver\\
    4 nypor&salt\\
    \SI{4}{dl} & ris\\
    lite&olja\medskip \\
    \multicolumn{2}{l}{\texttt{[frivilligt]}}\\
    lite&tabasco\\
    lite&chilipulver\\
    skvätt&Mango chutney\\
    skvätt&Sötsur sås\\
  }

  \preparation{%
    \step Koka riset enligt anvisningen på förpackningen.
    \step Hacka och stek löken i lagom mängd olja.
    \step Skär falukorven i strimlor och bryn tillsammans med löken.
    \step Häll ner tomaten, grädden, saltet och paprikapulvret. Smaksätt sedan, om man vill, med mango chutney eller sötsur sås. Lägg till lite hetta om så önskas.
    \step Blanda ut mjöl i vatten. Häll ner blandningen för att tjocka till stroganoffen.
    \step Servera tillsammans med riset.
  }
  \hint{Använd falukorv från Andersson \& Tillman för markant bästa effekt!}
\end{recipe}
