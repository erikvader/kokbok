\begin{recipe}[
  portion={\portion{5}},
  bakingtime={\SI{20}{\minute}},
  bakingtemperature={\SI{225}{\celsius}},
  preparationtime={\SI{60}{\minute}},
  source={\portionenundertian}
  ]{Vegetarisk Lasagne}
  \graph{big = \recipepicture{vegetarisk-lasagne}}

  \ingredients{%
    2 & gula lökar\\
    \SI{3}{\deci\liter} & torkade röda linser\\
    2$\times$\SI{400}{\gram} & krossade tomater\\
    \SI{6}{\deci\liter} & vatten\\
    3 & morötter\\
    2 & grönsaksbuljongtärningar\\
    \SI{2}{msk} & tomatpuré\\
    2 & vitlöksklyftor\\
    \SIrange{1}{2}{msk} & kryddor som oregano, timjan och basilika\\
    nypa & salt och peppar\\
    \SI{5}{dl} & crème fraiche\\
    \SI{2}{dl} & riven ost\\
    \SI{1}{pkt} & lasagneplattor\\
  }

  \preparation{%
    \step Börja med att göra tomatsåsen: hacka gul lök och stek med olja i kastrull tills löken är blank.
    \step Skölj linserna i kallt vatten. Lägg över i kastrullen. Slå på krossade tomater och vatten.
    \step Skölj och riv morötterna och lägg i. Tillsätt buljong och tomatpuré och låt koka under lock i \SI{10}{\minute} på medelsvag värme. Rör runt då och då.
    \step Späd eventuellt med mer vatten när såsen har kokat klart. Den ska vara tjock och inte rinning. Pressa i vitlöken. Smaka av med kryddor.
    \step Gör ostsås: blanda crème fraiche och ost. Osten ska smälta, men det skall ej koka. Spara ost till toppingen. Salta och peppra.
    \step Varva med lasagneplattor, ostsås och linssörjan i en långpanna. Grädda mitt i ugnen.
  }
\end{recipe}
